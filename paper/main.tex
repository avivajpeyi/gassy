% \documentclass[linenumbers,floatfix,ApJL,twocolumn]{aastex631}
\documentclass[floatfix,ApJL, twocolumn]{aastex631}

\usepackage{amssymb}
\usepackage{amsmath}
\usepackage{microtype}
\usepackage{url}
\usepackage{xspace}
\usepackage{xcolor}
\usepackage{ifxetex}
\ifxetex
\usepackage{fontspec}
\defaultfontfeatures{Extension = .otf}
\fi
\usepackage{fontawesome}



\setlength{\parindent}{3.0ex}


% Projects:
\newcommand{\project}[1]{\textsf{#1}}

\newcommand{\python}{\project{Python}}
\newcommand{\jupyter}{\project{Jupyter}}
\newcommand{\exoplanet}{\project{exoplanet}}
\newcommand{\lightkurve}{\project{lightkurve}}
\newcommand{\starry}{\project{starry}}
\newcommand{\pymc}{\project{PyMC3}}
\newcommand{\pymcextra}{\project{pymc3-ext}}
\newcommand{\celerite}{\project{celerite}}
\newcommand{\astropy}{\project{astropy}}
\newcommand{\scipy}{\project{scipy}}
\newcommand{\jupytext}{\project{jupytext}}
\newcommand{\sphinx}{\project{sphinx}}
\newcommand{\jupyterbook}{\project{Jupyter-book}}
\newcommand{\arviz}{\project{ArviZ}}
\newcommand{\nbconvert}{\project{nbconvert}}
\newcommand{\numpy}{\project{numpy}}
\newcommand{\pandas}{\project{pandas}}
\newcommand{\matplotlib}{\project{matplotlib}}
\newcommand{\corner}{\project{corner}}

\newcommand{\lvk}{\project{LVK}}
\newcommand{\tess}{\project{TESS}}
\newcommand{\mast}{\project{MAST}}
\newcommand{\exofop}{\project{ExoFOP}}


% math
\newcommand{\T}{\ensuremath{\mathrm{T}}}
\newcommand{\dd}{\ensuremath{ \mathrm{d}}}
\newcommand{\unit}[1]{{\ensuremath{ \mathrm{#1}}}}
\newcommand{\bvec}[1]{{\ensuremath{\boldsymbol{#1}}}}


\DeclareMathOperator{\invG}{Inv-\mathnormal{\Gamma}}
\DeclareMathOperator{\N}{\mathcal{N}}
\DeclareMathOperator{\U}{\mathcal{U}}
\DeclareMathOperator{\Un}{\mathcal{U}}
\DeclareMathOperator{\Par}{\mathcal{P}ar}
\DeclareMathOperator{\tmin}{\mathnormal{t_{\rm min}}}
\DeclareMathOperator{\tmax}{\mathnormal{t_{\rm max}}}





%% affiliation shortcuts
\newcommand{\SPA}{School of Physics and Astronomy, Monash University, Clayton VIC 3800, Australia}
\newcommand{\OzGravMonash}{OzGrav: The ARC Centre of Excellence for Gravitational Wave Discovery, Clayton VIC 3800, Australia}
\newcommand{\AMNH}{Department of Astrophysics, American Museum of Natural History, New York, NY 10024, USA}
\newcommand{\CCA}{Center for Computational Astrophysics, Flatiron Institute, New York, NY 10010, USA}
\newcommand{\CUNY}{Graduate Center, City University of New York, 365 5th Avenue, New York, NY 10016, USA}
\newcommand{\BMCC}{Department of Science, BMCC, City University of New York, New York, NY 10007, USA}





\newcommand{\projectUrl}{\href{http://google.com/}{some-link}}
\newcommand{\projectGit}{\href{https://github.com/avivajpeyi/gassy}{github.com/avivajpeyi/gassy}}

\newif\ifdraft
\drafttrue % switch to false in non-draft version (thereby hiding the todos)
\draftfalse
\newcommand{\inDraftVersion}[1]{\ifdraft #1\fi}


% TODOs
\newcommand{\todo}[3]{\inDraftVersion{{\color{#2}\emph{#1}: #3}}}
\newcommand{\avi}[1]{\todo{Avi}{red}{#1}}
\newcommand{\alltodo}[1]{\todo{TODO}{red}{#1}}
\newcommand{\citeme}{{\color{red}(citation needed)}}


\newcommand{\red}{\inDraftVersion{\textcolor{red}}}
\newcommand{\textuit}[1]{\textit{\underline{#1}}}


\newcommand{\github}[1]{\href{#1}{\textcolor{gray}{\faGithubSquare}}}








\shorttitle{Gassy}


\begin{document}

\title{LISA inference for a Red giant -- compact object common envelope binary Gravitational wave}

\shortauthors{Authors et al.}
\author[0000-0002-4146-1132]{Authors}%
\affiliation{\SPA}
\affiliation{\OzGravMonash}

% \shortauthors{Vajpeyi et al.}
% \author[0000-0002-4146-1132]{Avi Vajpeyi}%
% \affiliation{\SPA}
% \affiliation{\OzGravMonash}







\begin{abstract}
We present BLAH
\end{abstract}


\keywords{%
  methods: data analysis ---
  methods: statistical ---
  miscellaneous
}


\section{Relavent papers} 
\begin{itemize}
    \item \cite{Ginat:2020:MNRAS}: GW waveform from CE signals (final-dynamical phase of a CE)
    \item \cite{Renzo:2021:ApJ}: measuring GW from long-duration CE signals (looking at breaking index with pop-synth, no need for PE)
    \item \cite{Raveh:2021:MNRAS}: GW from CO-inspiraling into massive star
    \item \cite{Roepke:2022:arXiv}: review on simulations 
    \item \cite{Ohlmann:2016:ApJL}: moving mesh simulation with LISA SNR
    \item \cite{Wagg:2022:ApJ}: more pop-synth, can investigate CE 
\end{itemize}


\section{Questions about CE}
\begin{itemize}
    \item Which systems manage to eject the CE? Efficency $\alpha$? 
    \item What is the final orbital separation of the two stellar cores post-ejection?
\end{itemize}


\section{How can LISA help understand CE?}
\subsection{Direct observations of CE-GW}
Scenario with a giant red star (with compact core) primary doner and a CO secondary (close enough to inspiral).

Due to the inspiral of the secondary CO into the primary star’s envelope, orbital energy, and angular momentum are transferred to the gas. Some material becomes unbound and is ejected from the system. Simulations, however, show that this process alone is inefficient and other energy sources \cite[e.g. ionization of envelope material][]{Roepke:2022:arXiv} are required for full ejection (a successful CE). 
Failed CE ejection events may produce stronger GW signals (due to the plunge of CO). Measuring GW may help constrain some of the physics of the scenario. 


\paragraph{\textbf{(i)Late-Inspiral phase:}}
\cite{Ginat:2020:MNRAS} demos a parametrized description of CE inspiral,  predict about one detection in a \textbf{few centuries} with LISA. \cite{Ohlmann:2016:ApJL}'s simulations result in lower-SNR signals. 
\textbf{This makes measuring late-inspiral GW signals from LISA not so promising}.

\paragraph{\textbf{(ii)Earlier thermal-timescale phase:}}
\cite{Renzo:2021:ApJ} runs some pop-synth, demos rate of $\sim0.1-100$ events during LISA mission, some of which are detectable (based on how close the stellar cores come). 



\subsection{Indirect information from detecting post-CE sources}
Assuming all isolated-stellar mass LISA sources have undergone a CE phase, a comparison of LISA populations with model predictions (eg \cite{Wagg:2022:ApJ}) can test CE physics. 
LISA binary-population will be able to produce a detection rate as a function of orbital separation (after the CE phase). 



\section{Introduction}\label{sec:intro}

The remainder of the paper is organized as follows.
Section~\ref{sec:method} outlines...
Results are summarized in Section~\ref{sec:results}.
The data products and software to reproduce the results available online as supplementary materials (\projectUrl).
Finally, we discuss caveats and provide concluding remarks in Section~\ref{sec:conclusion}.




\section{Method} \label{sec:method}


\textbf{1. Data generation:} Generate a range of inspirals:

\begin{itemize}
    \item Check distribution of Red giant + CO (WD/NS) mass, radii, the initial and final separation that undergoes CE 
    \item Polytropic index of 1.5 
    \item Vary masses of WD (0.6)/ NS (1.4) and doner using mass (3, 12), radii (10, 200), and separations (edge of star)
    \item Inject into LISA noise
\end{itemize}

\textbf{2. Reduce order model:} Build waveform ROM

\textbf{3. LISA Injection study:} inject signal, try to recover some parameters 













\section{Results}\label{sec:results}
Details on the results

\section{Discussion and caveats}\label{sec:conclusion}
Some concluding discussions

\section{Data and software availability}\label{sec:data}
Where is my code/data?



%%%%%%%%%%%%%%%%%%%%



\section*{Acknowledgments}{


We gratefully acknowledge the Swinburne Supercomputing OzSTAR Facility for computational resources. All analyses (including test and failed analyses) performed for this study used $XX$K core hours on OzSTAR. This would have amounted to a carbon footprint of ${\sim XX{\text{t CO}_2}}$~\citep{greenhouse, energy_to_co2_converter}. Thankfully, as OzSTAR is powered by wind energy from Iberdrola Australia; the electricity for computations produces negligible carbon waste.


A.V. is supported by the Australian Research Council (ARC) Centre of Excellence CE170100004.

}

\vspace{5mm}
\facilities{\tess, \mast, Exoplanet Archive.}

\software{
\python~\citep{pythonForScientificComputing,pythonForScientists},
\astropy~\citep{astropy},
\arviz~\citep{arviz_2019},
\exoplanet~\citep{Foreman-Mackey:2021:JOSS},
\lightkurve~\citep{LightkurveCollaboration:2018:ascl},
\starry~\citep{Luger:2019:AJ},
\celerite~\citep{Foreman-Mackey:2017:ascl},
\pymc~\citep{Salvatier:2016:ascl},
\numpy~\citep{numpy},
\scipy~\citep{SciPy},
\pandas~\citep{pandas},
\matplotlib~\citep{matplotlib},
\corner~\citep{corner},
\sphinx~\citep{sphinx_doc},
\jupyter~\citep{jupyter},
\jupyterbook~\citep{jupyter_book}.
}


% ADS bibliography link
% https://ui.adsabs.harvard.edu/user/libraries/_DyLS4HbTY-eJIMBiFUdxw
\bibliography{main}{}
\bibliographystyle{aasjournal}

%%%%%%%%%%%%%%%%%%%%%%%%%%%%%%%%%%%%%%%%%%%%
\appendix

\section{Some More dets}\label{apdx:dets}
Some additional dets useful for the reader



\end{document}
